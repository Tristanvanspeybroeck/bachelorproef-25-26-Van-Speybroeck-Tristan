%==============================================================================
% Sjabloon onderzoeksvoorstel bachproef
%==============================================================================
% Gebaseerd op document class `hogent-article'
% zie <https://github.com/HoGentTIN/latex-hogent-article>

% Voor een voorstel in het Engels: voeg de documentclass-optie [english] toe.
% Let op: kan enkel na toestemming van de bachelorproefcoördinator!
\documentclass{hogent-article}

% Invoegen bibliografiebestand
\addbibresource{voorstel.bib}

% Informatie over de opleiding, het vak en soort opdracht
\studyprogramme{Professionele bachelor toegepaste informatica}
\course{Bachelorproef}
\assignmenttype{Onderzoeksvoorstel}
% Voor een voorstel in het Engels, haal de volgende 3 regels uit commentaar
% \studyprogramme{Bachelor of applied information technology}
% \course{Bachelor thesis}
% \assignmenttype{Research proposal}

\academicyear{2025-2026}

% TODO: Werktitel
\title{Automatische feedback op turnoefeningen via video-analyse ondersteund door Large Language Models}

% TODO: Studentnaam en emailadres invullen
\author{Tristan Van Speybroeck}
\email{tristan.vanspeybroeck@student.hogent.be}
\projectrepo{https://github.com/Tristanvanspeybroeck/bachelorproef-25-26-Van-Speybroeck-Tristan}

% TODO: Geef de co-promotor op
\supervisor[Co-promotor]{,\href{mailto:}{}}

% Binnen welke specialisatierichting uit 3TI situeert dit onderzoek zich?
% Kies uit deze lijst:
%
% - Mobile \& Enterprise development
% - AI \& Data Engineering
% - Functional \& Business Analysis
% - System \& Network Administrator
% - Mainframe Expert
% - Als het onderzoek niet past binnen een van deze domeinen specifieer je deze
%   zelf
%
\specialisation{AI \& Data Engineering}
\keywords{Large Language Models, Turnen, Video-analyse, Machine learning}

\begin{document}

\begin{abstract}
  Technische feedback tijdens turnoefeningen zoals de kip aan barre is in recreatieve en laag-competitieve trainingsgroepen moeilijk consistent te geven door grote groepsomvang en beperkte coaches. Omdat foutieve techniek zowel de sportieve ontwikkeling als blessurerisico beïnvloedt, wordt nagegaan in welke mate video-analyse in combinatie met Large Language Models kwalitatieve en bruikbare feedback kan opleveren. Het doel van dit onderzoek is een proof-of-concept te ontwikkelen dat videobeelden van de kip omzet naar pose-informatie, technische fouten herkent op basis van afgeleide bewegingskenmerken en deze vertaalt naar begrijpelijke feedback. De aanpak omvat dataverzameling en annotatie, pose-estimatie via markerloze systemen zoals MediaPipe of OenPose, foutendetectie met regels of eenvoudige classificatiemethoden, en generatieve feedback via LLM, gevolgd door een evaluatie door middel van objectieve meetwaarden en beoordeling door ervaren coaches. Er wordt verwacht dat het systeem in staat is om meerdere veelvoorkomende foutpatronen te identificeren en aanvullende, nuttige feedback te formuleren die coaches ondersteunt bij het begeleiden van grote groepen turnsters. Daarbij wordt er rekening gehouden met beperkingen van markerloze analyse en de variabele nauwkeurigheid van LLM-output. De resultaten kunnen inzicht bieden in de haalbaarheid van AI-ondersteunende feedback binnen turntrainingen en een basis vormen voor verdere verfijningen en toekomstige toepassingen. 
\end{abstract}

\tableofcontents

% De hoofdtekst van het voorstel zit in een apart bestand, zodat het makkelijk
% kan opgenomen worden in de bijlagen van de bachelorproef zelf.
%---------- Inleiding ---------------------------------------------------------

% TODO: Is dit voorstel gebaseerd op een paper van Research Methods die je
% vorig jaar hebt ingediend? Heb je daarbij eventueel samengewerkt met een
% andere student?
% Zo ja, haal dan de tekst hieronder uit commentaar en pas aan.

%\paragraph{Opmerking}

% Dit voorstel is gebaseerd op het onderzoeksvoorstel dat werd geschreven in het
% kader van het vak Research Methods dat ik (vorig/dit) academiejaar heb
% uitgewerkt (met medesturent VOORNAAM NAAM als mede-auteur).
% 

\section{Inleiding}%
\label{sec:inleiding}
Turnen omvat zware technische veeleisende oefeningen waarbij het lichaam volledig wordt belast~\autocite{ijms262210929}. In deze sport zijn een nauwkeurige uitvoering en een correcte lichaamshouding essentieel voor zowel prestatieverbetering als blessurepreventie. Vooral op recreatief en laag-com\-pe\-ti\-tief niveau spelen coaches een cruciale rol in het begeleiden van turnsters en het corrigeren van foutieve bewegingen. In deze context zijn trainingsgroepen veelal groot, waardoor het voor coaches en eventuele hulptrainers moeilijk wordt om iedere sporter tijdig en systematisch feedback te geven. 

Wanneer turnsters onvoldoende of niet tijdig feedback krijgen, bestaat het risico dat foutieve technieken zich inwerken. Uit onderzoek blijkt dat gebrekkige technische uitvoering een belangrijke risicofactor is voor blessures in het turnen~\autocite{ijms262210929}. Technische fouten die niet vroeg tijdig gecorrigeerd worden kan niet alleen sportieve ontwikkeling vertragen, maar ook de kans op blessures verhogen. Het gebrek aan continue, individuele feedback vormt daarom een concreet probleem binnen in recreatie- en lage-competitie toestelturen.

Het onderzoek vertrekt vanuit de situatie van een typische Vlaamse turnvereniging waar recreatieve en laag-competitieve dames-turngroepen worden begeleid. Binnen soortgelijke verenigingen ervaren coaches dat de groepsomvang en beperkte trainingstijd het moeilijk maken om elke turnster continu, individueel een correcte technische feedback te geven. 

In dit onderzoek wordt het probleem benaderd vanuit één specifieke turnoefening, de kip aan barre (vrouwentoestel). Deze oefening vereist een gecontroleerde schouderinzet, een gestrekte lichaamshouding en een technische correcte timing. In de praktijk blijken veel recreatieve en lager competitieve turnsters terugkerende fouten te maken, zoals onvoldoende actieve schouderdruk, te vroege heupopening, gebogen armen of een verkeerde timing in de zwaai. Door de grote groepsdynamiek en beperkte tijd per turnster kunnen coaches technische details niet altijd onmiddellijk opmerken.

Met de snelle evolutie van technologie en voor\-al met de opkomst van Large Language Models\linebreak(LLM's), ontstaat de mogelijkheid om automatische bewegingsfeedback toegankelijk te maken voor turntrainingen. Dit voorstel onderzoekt de haalbaarheid van een systeem dat video-opnames van de kip analyseert via pose-estimatie en deze bewegingsdata door een LLM laat interpreteren. De centrale onderzoeksvraag luidt als volgt: ``In hoeverre kunnen Large Language Models kwalitatieve en toepasbare feedback leveren op de uitvoering van de turnoefening kip aan barre via video-analyse?''

Het doel van dit onderzoek is het ontwikkelen en evalueren van een proof-of-concept (PoC) dat video-opnames van de kip analyseert, technische fouten detecteert en heldere feedback genereert die coaches kan ondersteunen in hun begeleiding van recreatieve en laag-competitieve turnsters. Dit PoC vormt een eerste verkenning van hoe moderne AI-technieken geïntegreerd kunnen worden in sporttechnische ondersteuning binnen een realistische trainingscontext.

\section{Literatuurstudie}%
\label{sec:literatuurstudie}

\subsection{Probleemstelling en motivatie}

In het turnen is de technische uitvoering van beweging cruciaal voor zowel prestatie als blessurepreventie. Onderzoek toont aan dat onjuiste techniek een van de belangrijkste causale factoren is van blessures bij turnsters~\autocite{ijms262210929}. Overbelasting en onvoldoende herstel versterken dit risico nog eens, waardoor vroegtijdige en nauwkeurige feedback op bewegingstechniek essentieel is. Tegelijkertijd blijkt dat coaches in de praktijk niet altijd voldoende en consistente feedback kunnen geven, de kwaliteit van coachfeedback hangt af van de persoonlijke kennis en ervaring van de coach~\autocite{Jeraj2015}.~\textcite{Jeraj2015} benadrukken dat juist die ``aangevulde feedback'' van de coach belangrijk is voor prestatieverbetering, maar ook dat deze voornamelijk steunen op de eigen ervaringskennis van de coach. Deze factoren samen maken het omzetten van juiste technische analyse naar bruikbare feedback uitdagend.

In de praktijk wordt in diverse sporten steeds meer gebruikgemaakt van technologieën om sporters te monitoren en begeleiden. Een recent onderzoek concludeerde dat draagbare sensoren en krachtplaten significant ingezet kunnen worden om trainingsbelasting en herstel te meten ter preventie van blessures~\autocite{Rebelo2023}. De sensoren kunnen waardevolle data leveren, maar dit onderzoek richt zich op video-gebaseerde bewegingsanalyse zonder extra sensoruitrusting, om een laagdrempel oplossing te ontwikkelen die in vele trainingsomgevingen toepasbaar is.

\subsection{Bewegingsanalyse: sensoren en vision-based systemen}

Sporttechnologie omvat twee benaderingen om beweging te analyseren. Ten eerste zijn er de klassieke draagbare apparaten: accelerometers, gyroscopen en krachtplaten die direct biomechanische data registreren.~\textcite{Rebelo2023} laat zien dat gelijkaardige draagbare technologieën, inclusief metingen als \textit{Player Load} en de \textit{acute chronic workload ratio}, veelgebruikte metrics zijn in studies naar trainingsaanpassingen en blessurepreventie. Soortgelijke systemen kunnen zeer nauwkeurige bewegings- en belastingdata leveren. Het minpunt is dat ze meestal een lastige setup en dure apparatuur vereisen (bijvoorbeeld meerdere sensoren op het lichaam of experimentele opstelling). Bovendien zijn ze minder flexibel voor spontaan gebruik tijdens routineoefeningen.

Een tweede benadering is beeldgebaseerde analyse. Hierbij worden camera's gebruikt om turnbewegingen vast te leggen, en algoritmes (bijvoorbeeld deep learning gebaseerde pose-estimatie) reconstrueren, en daaruit lichaamshouding en -beweging zonder fysieke markeringen. Volgens ~\textcite{Adlou2025} volgen deze systemen een proces waarbij beeldherkenning de \textit{body landmarks} in 2D detecteren en vervolgens een biomechanisch model gebruikt om 3D-bewegingsgegevens af te leiden. Voorbeelden van commerciële oplossingen (zoals Theia3D) en open-source tools (bv. OpenSim/DeepLabCut) tonen aan dat dit concept werkt. Echter is het belangrijk dat de nauwkeurigheid nog niet gelijkstaat aan traditionele marker-gebaseerde systemen, een validatie-studie toonde bijvoorbeeld hoekfouten van drie tot vijftien graden in het transversale as vlak voor heup- en kniehoek loopanalyse , terwijl afwijkingen in andere vlakken (bijvoorbeeld enkelrotaties) opliepen tot tientallen graden. Ook singlecamera oplossingen (bijvoorbeeld smartphonevideo) blijven achter bij multi-camera setups voor snelle, driedimensionale turn bewegingen. Samengevat biedt markeloze video-analyse dus een praktischer alternatief voor laboratoriumopstellingen, maar de betrouwbaarheid ervan is momenteel matig bij complexe turnelementen.~\autocite{Adlou2025} In dit onderzoek zal daarom zorgvuldig woden bekeken hoe we deze technieken, zoals OpenPose, MediaPipe of DeepLabCut kunnen toepassen en kalibreren voor turnen zonder sensoren, rekening houdend met hun huidige nauwkeurigheidslimieten.

\subsection{AI-gebaseerde methoden en multimodale modellen}

Naast beeldgebaseerde analyse komt kunstmatige intelligentie meer en meer om de hoek kijken. Convolutionele neurale netwerken (CNN’s) zijn bijvoorbeeld succesvol toegepast bij het detecteren van sportelementen. In UI-onderzoek werd YOLOv5 gebruikt voor het vinden van componenten in gebruikersinterface~\autocite{Altinbas2022}. In de sportanalyse zijn vergelijkbare netwerken aangepast om lichaamsdelen of sportgestel te herkennen. Voornamelijk \textit{pose-estimatie} netwerken kunnen rechtstreeks de posities van gewrichten bepalen in een video. Deze algoritmes (zoals HRNNet, OpenPose, Mediapipe Pose) kunnen op basis van voor getrainde modellen menselijke houding in kaart brengen. Door deze technieken kunnen training video's omgezet worden naar een ruwe hiërarchie van bewegende delen, die vervolgens gemeten worden op snelheid, hoek en coördinatie.

Tegelijkertijd is er een snelle ontwikkeling richting veelzijdige AI-modellen die beeld- en tekstinvoer combineren. OpenAI's CLIP-model kan bijvoorbeeld afbeeldingen en beschrijvingen ingebouwd in dezelfde vectorruimte en wordt in sommige studies gebruikt om overeenstemming tussen gegenereerde en echte beelden te meten~\autocite{Lee2025}. Nieuwere ``visuele taalmodellen'' (zoals GPT-4o Vision) kunnen volledige screenshots of opnames analyseren als input en tekstuele output generen. Dit opent perspectieven voor sporttoepassingen, men zou kunnen denken aan een systeem  dat videofragmenten van een oefening verwerkt en automatisch verbale of geschreven feedback formuleert. Hoewel concrete toepassingen nog experimenteel zijn, wijzen initiatieven in de sportwereld erop dat soortgelijke AI-coaches mogelijk bruikbaar zijn. ~\textcite{Lee2025} toonde recent aan dat een LLM periodiseerdeels schema's en oefenaanpassingen kon voorstellen en uitleg kon geven bij een halve marathon voorbereiding. Dit impliceert dat LLM's in staat zijn om complexe trainingsplannen aan te maken en simpele motiverende feedback te geven.

Echter zitten hier beperkingen aan. huidige LLM's kunnen geen directe sensordata verwerken en opteren vaak op tekstuele input/feedback. Uit de studie van ~\textcite{Lee2025} blijkt dat AI-coach zonder integratie van realtime gegevens (zoals hartslag of houding) geen directe correcties kon doen, en alleen via tekst communiceerde, wat subtiele vormcorrecties bemoeilijkt. Bovendien is de betrouwbaarheid nog een zorg. Nieuw onderzoek naar LLM's in sportgerelateerde taken (zoals voedingsadvies) laat zien dat de nauwkeurigheid vaak slechts matig is en sterk varieer per model~\autocite. In andere woorden, hoewel de technologie snel ontwikkelt, moet men kritisch blijven. LLM-output moet zorgvuldig gecontroleerd worden op feitelijke juistheid en volledigheid, zeker wanneer het om veiligheid en gezondheid gaat. In het algemeen kunnen we stellen dat geavanceerde AI-systemen veel beloven voor sportcoaching, maar dat de praktijk nu nog zwakheden kent in nauwkeurigheid en consistentie.~\autocite{Lee2025,Solomon2025}

\subsection{Samenvatting en vooruitblik}

Nog uit te werken.



%---------- Methodologie ------------------------------------------------------
\section{Methodologie}%
\label{sec:methodologie}

Hier beschrijf je hoe je van plan bent het onderzoek te voeren. Welke onderzoekstechniek ga je toepassen om elk van je onderzoeksvragen te beantwoorden? Gebruik je hiervoor literatuurstudie, interviews met belanghebbenden (bv.~voor requirements-analyse), experimenten, simulaties, vergelijkende studie, risico-analyse, PoC, \ldots?

Valt je onderwerp onder één van de typische soorten bachelorproeven die besproken zijn in de lessen Research Methods (bv.\ vergelijkende studie of risico-analyse)? Zorg er dan ook voor dat we duidelijk de verschillende stappen terug vinden die we verwachten in dit soort onderzoek!

Vermijd onderzoekstechnieken die geen objectieve, meetbare resultaten kunnen opleveren. Enquêtes, bijvoorbeeld, zijn voor een bachelorproef informatica meestal \textbf{niet geschikt}. De antwoorden zijn eerder meningen dan feiten en in de praktijk blijkt het ook bijzonder moeilijk om voldoende respondenten te vinden. Studenten die een enquête willen voeren, hebben meestal ook geen goede definitie van de populatie, waardoor ook niet kan aangetoond worden dat eventuele resultaten representatief zijn.

Uit dit onderdeel moet duidelijk naar voor komen dat je bachelorproef ook technisch voldoen\-de diepgang zal bevatten. Het zou niet kloppen als een bachelorproef informatica ook door bv.\ een student marketing zou kunnen uitgevoerd worden.

Je beschrijft ook al welke tools (hardware, software, diensten, \ldots) je denkt hiervoor te gebruiken of te ontwikkelen.

Probeer ook een tijdschatting te maken. Hoe lang zal je met elke fase van je onderzoek bezig zijn en wat zijn de concrete \emph{deliverables} in elke fase?

%---------- Verwachte resultaten ----------------------------------------------
\section{Verwacht resultaat, conclusie}%
\label{sec:verwachte_resultaten}

Hier beschrijf je welke resultaten je verwacht. Als je metingen en simulaties uitvoert, kan je hier al mock-ups maken van de grafieken samen met de verwachte conclusies. Benoem zeker al je assen en de onderdelen van de grafiek die je gaat gebruiken. Dit zorgt ervoor dat je concreet weet welk soort data je moet verzamelen en hoe je die moet meten.

Wat heeft de doelgroep van je onderzoek aan het resultaat? Op welke manier zorgt jouw bachelorproef voor een meerwaarde?

Hier beschrijf je wat je verwacht uit je onderzoek, met de motivatie waarom. Het is \textbf{niet} erg indien uit je onderzoek andere resultaten en conclusies vloeien dan dat je hier beschrijft: het is dan juist interessant om te onderzoeken waarom jouw hypothesen niet overeenkomen met de resultaten.



\printbibliography[heading=bibintoc]

\end{document}