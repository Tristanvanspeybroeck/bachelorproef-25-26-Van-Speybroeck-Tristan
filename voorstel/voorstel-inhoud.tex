%---------- Inleiding ---------------------------------------------------------

% TODO: Is dit voorstel gebaseerd op een paper van Research Methods die je
% vorig jaar hebt ingediend? Heb je daarbij eventueel samengewerkt met een
% andere student?
% Zo ja, haal dan de tekst hieronder uit commentaar en pas aan.

%\paragraph{Opmerking}

% Dit voorstel is gebaseerd op het onderzoeksvoorstel dat werd geschreven in het
% kader van het vak Research Methods dat ik (vorig/dit) academiejaar heb
% uitgewerkt (met medesturent VOORNAAM NAAM als mede-auteur).
% 

\section{Inleiding}%
\label{sec:inleiding}
Turnen omvat zware technische veeleisende oefeningen waarbij het lichaam volledig wordt belast~\autocite{ijms262210929}. In deze sport zijn een nauwkeurige uitvoering en een correcte lichaamshouding essentieel voor zowel prestatieverbetering als blessurepreventie. Vooral op recreatief en laag-com\-pe\-ti\-tief niveau spelen coaches een cruciale rol in het begeleiden van turnsters en het corrigeren van foutieve bewegingen. In deze context zijn trainingsgroepen veelal groot, waardoor het voor coaches en eventuele hulptrainers moeilijk wordt om iedere sporter tijdig en systematisch feedback te geven. 

Wanneer turnsters onvoldoende of niet tijdig feedback krijgen, bestaat het risico dat foutieve technieken zich inwerken. Uit onderzoek blijkt dat gebrekkige technische uitvoering een belangrijke risicofactor is voor blessures in het turnen~\autocite{ijms262210929}. Technische fouten die niet vroeg tijdig gecorrigeerd worden kan niet alleen sportieve ontwikkeling vertragen, maar ook de kans op blessures verhogen. Het gebrek aan continue, individuele feedback vormt daarom een concreet probleem binnen in recreatie- en lage-competitie toestelturen.

Het onderzoek vertrekt vanuit de situatie van een typische Vlaamse turnvereniging waar recreatieve en laag-competitieve dames-turngroepen worden begeleid. Binnen soortgelijke verenigingen ervaren coaches dat de groepsomvang en beperkte trainingstijd het moeilijk maken om elke turnster continu, individueel een correcte technische feedback te geven. 

In dit onderzoek wordt het probleem benaderd vanuit één specifieke turnoefening, de kip aan barre (vrouwentoestel). Deze oefening vereist een gecontroleerde schouderinzet, een gestrekte lichaamshouding en een technische correcte timing. In de praktijk blijken veel recreatieve en lager competitieve turnsters terugkerende fouten te maken, zoals onvoldoende actieve schouderdruk, te vroege heupopening, gebogen armen of een verkeerde timing in de zwaai. Door de grote groepsdynamiek en beperkte tijd per turnster kunnen coaches technische details niet altijd onmiddellijk opmerken.

Met de snelle evolutie van technologie en voor\-al met de opkomst van Large Language Models\linebreak(LLM's), ontstaat de mogelijkheid om automatische bewegingsfeedback toegankelijk te maken voor turntrainingen. Dit voorstel onderzoekt de haalbaarheid van een systeem dat video-opnames van de kip analyseert via pose-estimatie en deze bewegingsdata door een LLM laat interpreteren. De centrale onderzoeksvraag luidt als volgt: ``In hoeverre kunnen Large Language Models kwalitatieve en toepasbare feedback leveren op de uitvoering van de turnoefening kip aan barre via video-analyse?''

Het doel van dit onderzoek is het ontwikkelen en evalueren van een proof-of-concept (PoC) dat video-opnames van de kip analyseert, technische fouten detecteert en heldere feedback genereert die coaches kan ondersteunen in hun begeleiding van recreatieve en laag-competitieve turnsters. Dit PoC vormt een eerste verkenning van hoe moderne AI-technieken geïntegreerd kunnen worden in sporttechnische ondersteuning binnen een realistische trainingscontext.

\section{Literatuurstudie}%
\label{sec:literatuurstudie}

\subsection{Probleemstelling en motivatie}

In het turnen is de technische uitvoering van beweging cruciaal voor zowel prestatie als blessurepreventie. Onderzoek toont aan dat onjuiste techniek een van de belangrijkste causale factoren is van blessures bij turnsters~\autocite{ijms262210929}. Overbelasting en onvoldoende herstel versterken dit risico nog eens, waardoor vroegtijdige en nauwkeurige feedback op bewegingstechniek essentieel is. Tegelijkertijd blijkt dat coaches in de praktijk niet altijd voldoende en consistente feedback kunnen geven, de kwaliteit van coachfeedback hangt af van de persoonlijke kennis en ervaring van de coach~\autocite{Jeraj2015}.~\textcite{Jeraj2015} benadrukken dat juist die ``aangevulde feedback'' van de coach belangrijk is voor prestatieverbetering, maar ook dat deze voornamelijk steunen op de eigen ervaringskennis van de coach. Deze factoren samen maken het omzetten van juiste technische analyse naar bruikbare feedback uitdagend.

In de praktijk wordt in diverse sporten steeds meer gebruikgemaakt van technologieën om sporters te monitoren en begeleiden. Een recent onderzoek concludeerde dat draagbare sensoren en krachtplaten significant ingezet kunnen worden om trainingsbelasting en herstel te meten ter preventie van blessures~\autocite{Rebelo2023}. De sensoren kunnen waardevolle data leveren, maar dit onderzoek richt zich op video-gebaseerde bewegingsanalyse zonder extra sensoruitrusting, om een laagdrempel oplossing te ontwikkelen die in vele trainingsomgevingen toepasbaar is.

\subsection{Bewegingsanalyse: sensoren en vision-based systemen}

Sporttechnologie omvat twee benaderingen om beweging te analyseren. Ten eerste zijn er de klassieke draagbare apparaten: accelerometers, gyroscopen en krachtplaten die direct biomechanische data registreren.~\textcite{Rebelo2023} laat zien dat gelijkaardige draagbare technologieën, inclusief metingen als \textit{Player Load} en de \textit{acute chronic workload ratio}, veelgebruikte metrics zijn in studies naar trainingsaanpassingen en blessurepreventie. Soortgelijke systemen kunnen zeer nauwkeurige bewegings- en belastingdata leveren. Het minpunt is dat ze meestal een lastige setup en dure apparatuur vereisen (bijvoorbeeld meerdere sensoren op het lichaam of experimentele opstelling). Bovendien zijn ze minder flexibel voor spontaan gebruik tijdens routineoefeningen.

Een tweede benadering is beeldgebaseerde analyse. Hierbij worden camera's gebruikt om turnbewegingen vast te leggen, en algoritmes (bijvoorbeeld deep learning gebaseerde pose-estimatie) reconstrueren, en daaruit lichaamshouding en -beweging zonder fysieke markeringen. Volgens ~\textcite{Adlou2025} volgen deze systemen een proces waarbij beeldherkenning de \textit{body landmarks} in 2D detecteren en vervolgens een biomechanisch model gebruikt om 3D-bewegingsgegevens af te leiden. Voorbeelden van commerciële oplossingen (zoals Theia3D) en open-source tools (bv. OpenSim/DeepLabCut) tonen aan dat dit concept werkt. Echter is het belangrijk dat de nauwkeurigheid nog niet gelijkstaat aan traditionele marker-gebaseerde systemen, een validatie-studie toonde bijvoorbeeld hoekfouten van drie tot vijftien graden in het transversale as vlak voor heup- en kniehoek loopanalyse , terwijl afwijkingen in andere vlakken (bijvoorbeeld enkelrotaties) opliepen tot tientallen graden. Ook singlecamera oplossingen (bijvoorbeeld smartphonevideo) blijven achter bij multi-camera setups voor snelle, driedimensionale turn bewegingen. Samengevat biedt markeloze video-analyse dus een praktischer alternatief voor laboratoriumopstellingen, maar de betrouwbaarheid ervan is momenteel matig bij complexe turnelementen.~\autocite{Adlou2025} In dit onderzoek zal daarom zorgvuldig woden bekeken hoe we deze technieken, zoals OpenPose, MediaPipe of DeepLabCut kunnen toepassen en kalibreren voor turnen zonder sensoren, rekening houdend met hun huidige nauwkeurigheidslimieten.

\subsection{AI-gebaseerde methoden en multimodale modellen}

Naast beeldgebaseerde analyse komt kunstmatige intelligentie meer en meer om de hoek kijken. Convolutionele neurale netwerken (CNN’s) zijn bijvoorbeeld succesvol toegepast bij het detecteren van sportelementen. In UI-onderzoek werd YOLOv5 gebruikt voor het vinden van componenten in gebruikersinterface~\autocite{Altinbas2022}. In de sportanalyse zijn vergelijkbare netwerken aangepast om lichaamsdelen of sportgestel te herkennen. Voornamelijk \textit{pose-estimatie} netwerken kunnen rechtstreeks de posities van gewrichten bepalen in een video. Deze algoritmes (zoals HRNNet, OpenPose, Mediapipe Pose) kunnen op basis van voor getrainde modellen menselijke houding in kaart brengen. Door deze technieken kunnen training video's omgezet worden naar een ruwe hiërarchie van bewegende delen, die vervolgens gemeten worden op snelheid, hoek en coördinatie.

Tegelijkertijd is er een snelle ontwikkeling richting veelzijdige AI-modellen die beeld- en tekstinvoer combineren. OpenAI's CLIP-model kan bijvoorbeeld afbeeldingen en beschrijvingen ingebouwd in dezelfde vectorruimte en wordt in sommige studies gebruikt om overeenstemming tussen gegenereerde en echte beelden te meten~\autocite{Lee2025}. Nieuwere ``visuele taalmodellen'' (zoals GPT-4o Vision) kunnen volledige screenshots of opnames analyseren als input en tekstuele output generen. Dit opent perspectieven voor sporttoepassingen, men zou kunnen denken aan een systeem  dat videofragmenten van een oefening verwerkt en automatisch verbale of geschreven feedback formuleert. Hoewel concrete toepassingen nog experimenteel zijn, wijzen initiatieven in de sportwereld erop dat soortgelijke AI-coaches mogelijk bruikbaar zijn. ~\textcite{Lee2025} toonde recent aan dat een LLM periodiseerdeels schema's en oefenaanpassingen kon voorstellen en uitleg kon geven bij een halve marathon voorbereiding. Dit impliceert dat LLM's in staat zijn om complexe trainingsplannen aan te maken en simpele motiverende feedback te geven.

Echter zitten hier beperkingen aan. huidige LLM's kunnen geen directe sensordata verwerken en opteren vaak op tekstuele input/feedback. Uit de studie van ~\textcite{Lee2025} blijkt dat AI-coach zonder integratie van realtime gegevens (zoals hartslag of houding) geen directe correcties kon doen, en alleen via tekst communiceerde, wat subtiele vormcorrecties bemoeilijkt. Bovendien is de betrouwbaarheid nog een zorg. Nieuw onderzoek naar LLM's in sportgerelateerde taken (zoals voedingsadvies) laat zien dat de nauwkeurigheid vaak slechts matig is en varieert per model. In andere woorden, hoewel de technologie snel ontwikkelt, moet men kritisch blijven. LLM-output moet zorgvuldig gecontroleerd worden op feitelijke juistheid en volledigheid, zeker wanneer het om veiligheid en gezondheid gaat. In het algemeen kunnen we stellen dat geavanceerde AI-systemen veel beloven voor sportcoaching, maar dat de praktijk nu nog zwakheden kent in nauwkeurigheid en consistentie.~\autocite{Lee2025,Solomon2025}

\subsection{Samenvatting en vooruitblik}

Nog uit te werken.



%---------- Methodologie ------------------------------------------------------
\section{Methodologie}%
\label{sec:methodologie}

Het onderzoek volgt een stapsgewijze aanpak in meerde fasen, van dataverzameling tot evaluatie. In Fase 1 verzamelen we videobeelden van de kip-oefening en labelen we relevante gebeurtenissen. In Fase 2 gebruiken we pose-estimatie-algoritmes om lichaamsskeletten uit de video af te leiden. Tijdens Fase 3 analyseren we deze poses om veelvoorkomende technische fouten te detecteren. In Fase 4 wordt op basis van de gedetecteerde fouten   begrijpelijke taal gegenereerd via LLM (bijv. GPT-4) om feedback te formuleren. Ten slotte volgt in Fase 5 een grondige evaluatie van zowel de foutendetectie als kwaliteit van de de gegenereerde feedback, waarna het systeem iteratief wordt verbeterd, Hieronder worden deze fasen in detail beschreven, inclusief de gebruikte methoden, tools en geschatte tijdsplanning.

\subsection{Fase 1: Dataverzameling en annotatie}

In deze fase wordt een trainings- en testset aan met een video-opnames van de kip aan de barre vast gelegd. Dit gaat als volg te werk:

\begin{itemize}
    \item \textbf{Video-opnames}: Tijdens trainingssessies bij de turnvereniging worden video-opnames gemaakt van sporters die de kip-oefening uitvoeren, zowel in correcte als foutieve vorm. Indien beschikbaar zal er gebruik gemaakt worden van bestaande videomateriaal. Er zal aandacht besteed worden op variatie in uitvoerders en hoeken om generaliseerbaarheid te vergroten.
    \item \textbf{Annotatie van gebeurtenissen}: In de verzamelde video's worden de momenten gemarkeerd in de oefening. Voor elke opname word er genoteerd of er fouttypes voorkomen (bijv. te vroege heupopening). Er gaat gebruik gemaakt worden van eenvoudige annotatietools of Python-scripts om belangrijke tijdstippen en labels in een gestructureerd (bv. JSON) formaat op te slaan.
    \item \textbf{Dataset-samenstelling}: Het resultaat is een dataset van enkele tientallen tot honderd gelabelde kip opnames. Deze dataset wordt gesplitst in trainings-en validatiesets. Het doel is een representatieve verzameling video's met bijhorende foutlabels en relevante metadata op te bouwen, Dit biedt de basis voor de volgende fasen
\end{itemize}

De doorlooptijd voor Fase 1 wordt geschat op ongeveer 2-3 weken. De concrete deliverables zijn een geannoteerde videodatabase en bijhorende labelbestanden.

\subsection{Fase 2: Pose-estimatie en bewegingsextractie}

In deze fase worden uit de video-opnames de lichaamshouding gehaald in de vorm van gewrichtscoördinaten. Voor deze taak worden populaire markerloze frameworks zoals MediaPipe Pose en OpenPose overwogen, die in de sportanalyse veel worden toegepast. MediaPipe biedt een gebruiksvriendelijke API en is effectief gebleken voor real-time lichaamsdetectie in sportomgeving~\autocite{Lyu2025}, terwijl OpenPose een volwassen en robuust alternatief vormt voor 2D-pose-estimatie. De volgende stappen worden doorlopen:

\begin{itemize}
    \item \textbf{ Pose-tracking implementatie}: Met Python-tools (bijv. OpenCV) word frame-voor-frame pose-estimatie uitgevoerd. Bijvoorbeeld wordt voor elke videoframe een skelet gegenereerd met 2D-coördinaten van hoof- en ledemaat punten (schouders, heupen, knieën, enkels, enz.). Dit kan met MediaPipe Pose of OpenPose. De videobestanden worden scriptmatig behandeld, zodat per frame de gewrichtsposities en bijhorende vertrouwensscores opslaan kunnen worden.
    \item \textbf{Validatie van Poses}: De output word gecontroleerd op kwaliteit. Een eerder onderzoek toont aan dat dergelijke markerloze systemen enige onnauwkeurigheid hebben, vaak worden systematische afwijkingen waargenomen (bijv. ~30-50 mm bij heup/knee) vergeleken met lab-sensoren~\autocite{Needham2021}. Desondanks is de verkregen bewegingsschatting voldoende voor het beoordelen van houdingspatronen. De camera-instellingen kunnen eventueel afgestemd worden om perspectief te beperken.
    \item \textbf{Feature-extractie}: Uit de gemeten gewrichtsposities worden relevante afgeleiden berekend, zoals gewrichtshoeken en hoeksnelheden op cruciale momenten. Deze tijdreeksen van kenmerken vormen vervolgens de input voor foutanalyse in de volgende fase
\end{itemize}

De geschatte duur van deze fase is ongeveer 3-4 weken. De deliverable omvat een geautomatiseerde pipeline die van elke videofragment de pose-gegevens exporteert en scripts die deze data voorbereiden voor analyse.

\subsection{Fase 3: technische analyse en foutdetectie}

In deze fase worden de verworven poses op aanwezigheid van technische fouten. Voor coaches zijn typische fouten voor de kip oefening bekend. We voeren de volgende analyses uit:

\begin{itemize}
    \item \textbf{Kenmerken en regels}: Er worden kwantitatieve regels voor de belangrijkste fouttypes. Bijvoorbeeld een correcte kip vereist een bepaalde romphoek, volledige armextensie  en voldoende schouderflexie. Deze voorwaarden vertalen worden vertaalt naar drempelwaarden voor de gewrichtshoeken op specifieke tijdstippen.
    \item  \textbf{Foutclassificatie}: Met de regels of met begeleide leermethoden wordt gecontroleerd voor elke oefening of conditie dat er voldaan wordt. Mogelijks kan er een eenvoudige classifier (bijv. een SVM) op de gelabelde dataset om vlagen of foutcategorieën te onderscheiden. Daarbij kan gebruik worden gemaakt van algoritmes zoals Dynamic Time Warping, dat is toegepast in soortgelijke studies om bewegingstrajecten te vergelijken~\autocite{Lyu2025}. Zo zal er vergeleken worden op het verloop van de gewrichtshoeken met een ideaalprofiel en markeren afwijkingen als fouten.
    \item \textbf{Tooling}: Er zal gebruik gemaakt worden van Python-libraries zoals NumPy en scikit-learn voor berekeningen en classificaties. De input zijn de pose-features uit fase 2 en eventueel extra annotaties. Het resultaat per video is een lijst van gedetecteerde fouten (type + tijdstip).
\end{itemize}

De duur voor Fase 3 wordt geraamd op ongeveer 4 weken. De deliverables zijn software-scripts voor foutdetectie en een beschrijving van de gedetecteerde foutpatronen per scherm.

\subsection{Fase 4: Generatieve feedback via een LLM}

Nadat de technische gouten zijn gelokaliseerd, worden heldere toepasbare feedback gegenereerd in duidelijke taal. Hiervoor zetten we een taalmodel in (bijv. GPT-4 via de OpenAI API). De workflow is als volgt:

\begin{itemize}
    \item \textbf{Prompt engineering}: Voor elke video-oefening word een tekstprompt opgebouwd waarin de gevonden fouten en context beschrijven.
    \item \textbf{Feedbackgeneratie}: Het LLm krijgt de geannoteerde oefeningstekst (bijv. ``De schouders zijn niet genoeg doorgedrukt tijdens de zwaai'') als input en genereert daarop een advieszin of -paragraaf. Er word geëxperimenteerd met verschillende formuleringen en formats, en vragen het model om in het Nederlands feedback te geven.
    \item \textbf{Kwaliteitscontrole}: Uit studies blijkt dat LLM-gegenereerde adviezen nuttig kunnen zijn voor planning en uitleg, maar momenteel vooral tekst-only zijn en geen directe sensorische context gebruiken~\autocite{Lee2025}. Hierdoor word de output kritisch beoordeeld of het correct en bruikbaar is. Bij onduidelijkheden wirdt feedback manueel bijgesteld.
\end{itemize}

De duur voor Fase 4 is ongeveer 3 weken, met als deliverables de gegenereerde feedbacktesten per oefening en documentatie van de prompt samenstelling.

\subsection{Fase 5: Evaluatie en iteratie}

In de laatste fase worden de prestaties beoordeeld van het volledige systeem en voeren we verbeteringen door. Dit verloopt in meerder cycli:

\begin{itemize}
    \item \textbf{Evaluatie van detectie en feedback}: Er word objectief gemeten op de foutdetectie-kwaliteit met meetwaarden zoals precisie en recall (bijv. F1-score) op een aparte testset. Daarnaast zullen ervaren coaches de gegenereerde feedback beoordelen op correctheid en bruikbaarheid.
    \item \textbf{Iteratieve verbeteringen}: Op basis van de evaluatieresultaten zal het systeem aangepast worden. Dit kan beteken dat de dataset word uitgebreid met extra videomateriaal, de regels of classifier voor foutdetectie verfijnen, of de prompts en voorbeeldvarianten voor de LLM bijsturen. 
\end{itemize}

De initiële evaluatiecyclus neemt naar verwachting 2-3 weken in beslag. De verwachte uitkomsten zijn een meetbare verbetering in detectieprecisie en meer accurate, consistentere feedback in opeenvolgende versies.


%---------- Verwachte resultaten ----------------------------------------------
\section{Verwacht resultaat, conclusie}%
\label{sec:verwachte_resultaten}

Er word verwacht dat het PoC-systeem meerdere veelvoorkomende technische fouten bij de kip kan detecteren en daarvoor bruikbare feedback genereert. Dit kan blijken uit een vergelijkingstabel of grafiek waarin de door coaches geïdentificeerde fouten staan tegenover het systeem. Bijvoorbeeld kan een balkdiagram aantonen dat het systeem een groot deel van de schouderfouten en arm positie fouten correct signaleert, in lijn met de coachobservaties. Hoewel exacte voorspellingen lastig, is het aannemelijk dat de genereerde feedback in grote lijnen overeenkomt met wat ervaren coaches zouden zeggen.

Voor de doelgroep (trainers en turnsters) zit de meerwaarde vooral in extra ondersteuning. Het systeem kan dienen als een tweede oog, in drukke groepen vangt het details op die een coach op dat moment misschien mist of indien die persoon bezig is met een andere turnster. Hierdoor krijgen turnsters sneller gerichte tips, wat hun vooruitgang en blessurepreventie ten goede komt. In de praktijk zou een coach bijvoorbeeld na een oefening de video via het systeem kunnen laten lopen en zo onmiddellijke, objectieve feedback krijgen, naast zijn eigen observaties. Dit kan leiden tot efficiënter trainen en eventuele technische achterstanden sneller corrigeren.

Doch worden er beperkingen verwacht. Het is voorspelbaar dat de AI-feedback niet foutloos is. In de sportcontext blijkt uit onderzoek dat generatieve LLM-adviezen nu vaak slechts matig accuraat zijn en per model verschillen. Dit betekent dat sommige adviezen oppervlakkig of te algemeen kunnen blijven. In de resultaten van het onderzoek zullen dit zich uiten in verschillen tussen coach- en AI-feedback. Zulke afwijkingen kunnen bijvoorbeeld in gevallen voorkomen waar een subtiel verschil niet goed door het model wordt opgenomen, of waar het model een onnodige detail-instructie geeft. Dit is geen falen, maar een waardevolle uitkomst, het laat zien op welke gebieden LLM's nog te kort schieten en welke verbeteringen nodig zijn. Als we merken dat het systeem consistent dezelfde zwakke punten heeft, vormt dat een uitgangspunt voor verdere verfijning.

In ons PoC wordt er alleen gebruik gemaakt van standaard camera's en AI-software. Uit de literatuur blijkt dat draagbare sensoren en marker-gebaseerde systemen de bewegingsdata veel nauwkeuriger kan vastleggen. Deze meten direct gewrichtshoeken en krachten, en zijn de ``gouden standaard'' in fysiomechanica. Er mag verwacht worden dat deze soort opstellingen een hogere meetnauwkeurigheid geven. Tegelijkertijd blijkt dat een gecombineerde opstelling 3D-locaties met hoge precisie kan reconstrueren. Dit suggereert dat ons systeem weliswaar praktisch en goedkoop is, maar dat de precisie lager ligt. Er wordt een algemene nauwkeurigheid verwacht maar niet met dezelfde precisie als met een sensorgebaseerd systeem. Indien er uit de experimenten blijken dat de output te onnauwkeurige is, dan kan er worden voorgesteld dat toekomstige studies beter gebruik maken van sensoren of meerdere camera's om de feedbackkwaliteit te verbeteren.

In conclusie wordt er verwacht dat het onderzoek aantoont dat moderne AI-technieken veelbelovende aanvullingen kunnen leveren in sportcoaching, maar ook dar ze grenzen hebben. Het PoC zal vermoedelijk bruikbare, zij het niet perfecte, feedback opleveren. Het resultaat voor de doelgroep is een prototype dat laat zien hoe coaches ondersteund kunnen worden. Als de uitkomsten anders zijn dan de hypothesen, dan belicht het waarom en welke factoren daarbij een rol spelen. In alle gevallen zal het een waardevolle inzichten over de haalbaarheid van AI-gebaseerd hulpmiddel voor turntrainingen leveren, en vormt een basis voor verdere verbeteringen in deze richting.

